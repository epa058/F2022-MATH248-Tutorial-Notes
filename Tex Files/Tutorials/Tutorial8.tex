\subsection{Section 7.5}

\begin{tcolorbox}[
        title={Problem 6},
        valign=center,
        nobeforeafter,
        colframe=gray!95!black
    ]
    Evaluate the integral:
    \begin{align}
        \iint_S x^2z + y^2z \ dS
    \end{align}
    where \(S\) is the part of the plane \(z = 4 + x + y\) that lies inside the cylinder \(x^2 + y^2 = 4\).
\end{tcolorbox}

\begin{solution}
    The plane is restricted to inside of a cylinder. The surface \(S\) is then best parametrized in cylindrical coordinates. 
    
    Recall cylindrical coordinates:
    \begin{align}
        x &= r\cos(\theta) & y &= r\sin(\theta) & z &= z
    \end{align}
    
    The plane is then given by:
    \begin{align*}
        z &= 4 + x + y \\
        &= 4 + r \cos(\theta) + r \sin(\theta)
    \end{align*}
    
    The domain of integration is restricted to \(D = \{(x, y) \in \mathbb{R}^2 \ | \ x^2 + y^2 \leq 2^2\}\):
    \begin{align*}
        x^2 + y^2 &\leq 2^2 \\
        r^2 &\leq 2^2 \\
        r &\leq 2
    \end{align*}
    
    We therefore have the following parametrization for the surface \(S\):
    \begin{align}
        \Phi(r, \theta) &= \left(x, y, z\right) \\
        &= \left(r\cos(\theta), r\sin(\theta), 4 + r\cos(\theta) + r\sin(\theta)\right)
    \end{align}
    where \(0 \leq r \leq 2\) and \(0 \leq \theta \leq 2\pi\).
    
    We first compute the partial derivatives:
    \begin{align*}
        \Phi_r &= \left(\cos(\theta), \sin(\theta), \cos(\theta) + \sin(\theta)\right) & \Phi_\theta &= \left(-r\sin(\theta), r\cos(\theta), -r\sin(\theta) + r\cos(\theta)\right)
    \end{align*}
    
    We now compute the Jacobian:
    \begin{align*}
        \|\Phi_r \times \Phi_\theta\| &= \|\left(\cos(\theta), \sin(\theta), \cos(\theta) + \sin(\theta)\right) \times \left(-r\sin(\theta), r\cos(\theta), -r\sin(\theta) + r\cos(\theta)\right) \| \\
        &= \|\left(- r\sin^2(\theta) - r\cos^2(\theta), - r\sin^2(\theta) - r\cos^2(\theta), r\cos^2(\theta) + r\sin^2(\theta)\right) \| \\
        &= \|\left(- r, - r, r\right) \| \\
        &= \sqrt{r^2 + r^2 + r^2} \\
        &= r\sqrt{3}
    \end{align*}
    
    We now evaluate the integral:
    \begin{align*}
        \iint_S x^2z + y^2z \ dS &= \iint_S (x^2 + y^2)z \ dS \\
        &= \iint_D r^2(4 + r\cos(\theta) + r\sin(\theta)) \|\Phi_r \times \Phi_\theta\| \ dr d\theta \\
        &= \iint_D r^2(4 + r\cos(\theta) + r\sin(\theta)) r\sqrt{3} \ dr d\theta \\
        &= \sqrt{3}\int_0^{2\pi} \int_0^2 4r^3 + r^4\left(\cos(\theta) + \sin(\theta)\right) \ drd\theta \\
        &= \sqrt{3}\int_0^{2\pi} r^4\Big|_0^2 + \frac{r^5}{5}\left(\cos(\theta) + \sin(\theta)\right)\Biggr|_0^2 \ d\theta \\
        &= \sqrt{3}\int_0^{2\pi} 16 + \frac{32}{5}\left(\cos(\theta) + \sin(\theta)\right) \ d\theta \\
        &= \sqrt{3}\left( 16 \theta + \frac{32}{5}\left(\sin(\theta) - \cos(\theta)\right)\right) \Biggr|_0^{2\pi} \\\
        &= \sqrt{3}\left( 16 (2\pi)\right) \\
        &= 32 \pi \sqrt{3}
    \end{align*}
\end{solution}

\begin{tcolorbox}[
        title={Problem 17},
        valign=center,
        nobeforeafter,
        colframe=gray!95!black
    ]
    Let \(S\) be a sphere of radius \(R\). \\
    
    Argue by symmetry that:
    \begin{align}
        \iint_S x^2 \ dS = \iint_S y^2 \ dS = \iint_S z^2 \ dS
    \end{align}
    
    Use this fact to evaluate, with very little computation, the integral:
    \begin{align}
        \iint_S x^2 \ dS
    \end{align}
\end{tcolorbox}

\begin{solution}
    Since spheres are symmetric with respect to the \(xy\), \(xz\), and \(yz\) planes, interchanging the variables \(x\), \(y\), and \(z\) in the integral will not affect the value of the integral.
    
    We now evaluate:
    \begin{align}
        \iint_S x^2 \ dS
    \end{align}
    
    Recall spherical coordinates on a sphere of radius \(R\):
    \begin{align}
        x &= R\cos(\theta) \sin(\varphi) & y &= R\sin(\theta) \sin(\varphi) & z &= R\cos(\varphi) & x^2 + y^2 + z^2 &= R^2
    \end{align}
    
    Observe that:
    \begin{align*}
        \iint_S x^2 \ dS &= \iint_S \frac{x^2 + x^2 + x^2}{3} \ dS \\
        &= \iint_S \frac{x^2 + y^2 + z^2}{3} \ dS \\
        &= \frac{1}{3} \iint_S x^2 + y^2 + z^2 \ dS \\
        &= \frac{1}{3} \iint_S R^2 \ dS \\
        &= \frac{R^2}{3} \iint_S dS \\
        &= \frac{R^2}{3} (4\pi R^2) \\
        &= \frac{4\pi R^4}{3}
    \end{align*}
    where we used the fact that the total surface area of a sphere of radius \(R\) is equal to \(4\pi R^2\). 
\end{solution}

\subsection{Section 7.6}

\begin{tcolorbox}[
        title={Problem 9},
        valign=center,
        nobeforeafter,
        colframe=gray!95!black
    ]
    Let \(\vb{F} = y\vb{i} - x\vb{j} + x^3y^2z\vb{k}\). \\
    
    Evaluate:
    \begin{align}
        \iint_S (\nabla \times \vb{F}) \cdot d\vb{S}
    \end{align}
    where \(S\) is the surface \(x^2 + y^2 + 3z^2 = 1\) and \(z \leq 0\), oriented by the upward-pointing normal.
\end{tcolorbox}

\begin{solution}
    \textit{(The long way around)}

    Let \(D\) denote the unit disk:
    \begin{align}
        D &= \{(x, y) \in \mathbb{R}^2 \ | \ x^2 + y^2 \leq 1\}
    \end{align}
    
    Observe that, since the surface \(S\) lies below the \(xy\) plane, it can be parametrized as a graph \(z = g(x, y)\) over \(D\) given by:
    \begin{align}
        g(x, y) &= - \sqrt{\frac{1 - x^2 - y^2}{3}}
    \end{align}

    We first compute the partial derivatives of \(g\):
    \begin{align*}
        \frac{\partial g}{\partial x} &= \frac{x}{\sqrt{3(1 - x^2 - y^2)}} & \frac{\partial g}{\partial y} &= \frac{y}{\sqrt{3(1 - x^2 - y^2)}}
    \end{align*}

    We now calculate the curl of \(\vb{F}\):
    \begin{align*}
        \nabla \times \vb{F} &= 
        \begin{vmatrix}
            \vb{i} & \vb{j} & \vb{k} \\
            \frac{\partial}{\partial x} & \frac{\partial}{\partial y} & \frac{\partial}{\partial z} \\
            y & -x & zx^3y^2
        \end{vmatrix} \\
        &= \left(\frac{\partial (x^3y^2z)}{\partial y} - \frac{\partial (-x)}{\partial z}\right)\vb{i} + \left(\frac{\partial (y)}{\partial z} - \frac{\partial (x^3y^2z)}{\partial x}\right)\vb{j} + \left(\frac{\partial (-x)}{\partial x} - \frac{\partial (y)}{\partial y}\right)\vb{k} \\
        &= 2x^3yz\vb{i} - 3x^2y^2z\vb{j} -2\vb{k}
    \end{align*}

    We now evaluate the integral with \(z = g(x, y)\):
    \begin{align*}
        \iint_S \nabla \times \vb{F} \cdot d\vb{S} &= \iint_D \nabla \times \vb{F} \cdot (-g_x, -g_y, 1) \ dxdy \\
        &= \iint_D \left(2x^3yz, - 3x^2y^2z, -2\right) \cdot (-g_x, -g_y, 1) \ dxdy \\
        &= \iint_D -2x^3y z g_x + 3x^2y^2 z g_y -2 \ dxdy \\
        &= \iint_D 2x^3y \sqrt{\frac{1 - x^2 - y^2}{3}} \frac{x}{\sqrt{3(1 - x^2 - y^2)}} \\
        &\quad - 3x^2y^2 \sqrt{\frac{1 - x^2 - y^2}{3}} \frac{y}{\sqrt{3(1 - x^2 - y^2)}} -2 \ dxdy \\
        &= \iint_D \frac{2}{3}x^4y - x^2y^3 - 2 \ dxdy
    \end{align*}
    
    Now, recall polar coordinates:
    \begin{align}
        x &= r\cos(\theta) & y &= r\sin(\theta) & dxdy &= r \ dr d\theta
    \end{align}
    
    Then the integral is given by:
    \begin{align*}
        \iint_S \nabla \times \vb{F} \cdot d\vb{S} &= \iint_D \frac{2}{3}x^4y - x^2y^3 - 2 \ dxdy \\
        &= \int_0^1 \int_0^{2\pi} \left(\frac{2}{3}\left(r\cos(\theta)\right)^4\left(r\sin(\theta)\right) - \left(r\cos(\theta)\right)^2\left(r\sin(\theta)\right)^3 - 2\right) r \ d\theta dr \\
        &= \int_0^1 \int_0^{2\pi} \frac{2}{3}r^6\cos^4(\theta)\sin(\theta) - r^6\cos^2(\theta)\sin^3(\theta) - 2r \ d\theta dr \\
        &= \int_0^1 \int_0^{2\pi} \frac{2}{3}r^6\cos^4(\theta)\sin(\theta) \ d\theta dr - \int_0^1 \int_0^{2\pi} r^6\cos^2(\theta)\sin^3(\theta) \ d\theta dr \\
        &\quad - 2 \int_0^1 \int_0^{2\pi} r \ d\theta dr
    \end{align*}
    
    To solve this integral faster, we make use of an important identity. For all \(n\), \(m \in \mathbb{Z}\):
    \begin{align}
        \int_0^{2\pi} \sin^{2m+1}(x)\cos^n(x) \ dx &= 0 & \int_0^{2\pi} \sin^{m}(x)\cos^{2n+1}(x) \ dx &= 0
    \end{align}
    
    This follows from the fact that we are integrating over a full period. You should convince yourself that this makes sense before you use it.
    
    Then:
    \begin{align*}
        \iint_S \nabla \times \vb{F} \cdot d\vb{S} &= \int_0^1 \int_0^{2\pi} \frac{2}{3}r^6\cos^4(\theta)\sin(\theta) \ d\theta dr - \int_0^1 \int_0^{2\pi} r^6\cos^2(\theta)\sin^3(\theta) \ d\theta dr \\
        &\quad - 2 \int_0^1 \int_0^{2\pi} r \ d\theta dr \\
        &= - 2 \int_0^1 \int_0^{2\pi} r \ d\theta dr \\
        &= -4\pi \int_0^1 r \ dr \\
        &= -4\pi \frac{r^2}{2}\Biggr|_0^1 \\
        &= -2\pi 
    \end{align*}
\end{solution}

\begin{solution}
    \textit{(Stokes' Theorem)}

    You can visualize the surface \(S\) as a bowl lying below the \(xy\) plane. 
    
    Observe then that the boundary of the surface \(S\) is the unit circle lying on the \(xy\) plane:
    \begin{align}
        \partial S &= \{(x, y, z) \in \mathbb{R}^3 \ | \ x^2 + y^2 = 1, z = 0\}
    \end{align}

    Since the normal of the surface \(S\) points upwards, the orientation for this curve must be counter-clockwise.

    We may then parametrize this boundary by:
    \begin{align}
        \vb{c}(t) &= (\cos(t), \sin(t), 0)
    \end{align}
    where \(0 \leq t \leq 2\pi\).
    
    Observe that \(\vb{F}\) has no singularities on \(S\).
    
    Then by Stokes' Theorem:
    \begin{align*}
        \iint_S \nabla \times \vb{F} \cdot d \vb{S} &= \oint_{\partial S} \vb{F} \cdot d\vb{s} \\
        &= \int_{0}^{2\pi} \vb{F}(\vb{c}(t)) \cdot \vb{c}'(t) \ dt \\
        &= \int_{0}^{2\pi} (\sin(t), -\cos(t), 0) \cdot (-\sin(t), \cos(t), 0) \ dt \\
        &= \int_{0}^{2\pi} -\sin^2(t) - \cos^2(t) \ dt \\
        &= -\int_{0}^{2\pi} dt \\
        &= -2\pi
    \end{align*}
\end{solution}

\begin{tcolorbox}[
        title={Problem 11},
        valign=center,
        nobeforeafter,
        colframe=gray!95!black
    ]
    Let \(\vb{F} = (x + 3y^5) \vb{i} + (y + 10xz) \vb{j} + (z - xy) \vb{k}\). \\
    
    Calculate the integral:
    \begin{align}
        \iint_S \vb{F} \cdot d\vb{S}
    \end{align}
    where \(S\) is the entire surface of the solid half ball \(x^2 + y^2 + z^2 \leq 1\) and \(z \geq 0\), oriented by the outward-pointing normal (including the \(z = 0\) disk).
\end{tcolorbox}

\begin{solution}
    \textit{(The long way around)} 
    
    We split the surface integral into two parts: the flux of \(\vb{F}\) across the upper hemisphere \(S_1\) and the flux of \(\vb{F}\) across the disk \(S_2\).
    \begin{align}
        \iint_S \vb{F} \cdot d\vb{S} &= \iint_{S_1} \vb{F} \cdot d\vb{S} + \iint_{S_2} \vb{F} \cdot d\vb{S}
    \end{align}
    
    First, recall spherical coordinates on a unit sphere (\(R = 1\)):
    \begin{align}
        x &= \cos(\theta) \sin(\varphi) & y &= \sin(\theta) \sin(\varphi) & z &= \cos(\varphi) & dS &= \sin(\varphi) \ d\theta d\varphi
    \end{align}
    \begin{align}
        x^2 + y^2 + z^2 = 1^2
    \end{align}
    
    Then the flux across the upper hemisphere is given by:
    \begin{align*}
        \iint_{S_1} \vb{F} \cdot d\vb{S} &= \iint_{S_1} \vb{F} \cdot \vb{\hat{n}} \ dS \\
        &= \iint_{S_1} \vb{F} \cdot \vb{\hat{r}} \ dS \\
        &= \iint_{S_1} \left(x + 3y^5, y + 10xz, z - xy\right) \cdot \left(x, y, z\right) \ dS \\
        &= \iint_{S_1} x^2 + 3xy^5 + y^2 + 10xyz + z^2 - xyz \ dS \\
        &= \iint_{S_1} x^2 + y^2 + z^2 + 3xy^5 + 9xyz \ dS \\
    \end{align*}
    
    Since the surface \(S_1\) has a radius of 1, we may simplify \(x^2 + y^2 + z^2 = 1\):
    \begin{align*}
        \iint_{S_1} \vb{F} \cdot d\vb{S} &= \iint_{S_1} 1 + 3xy^5 + 9xyz \ dS \\
        &= \iint_{S_1} 1 + 3\cos(\theta) \sin(\varphi)\sin^5(\theta) \sin^5(\varphi) + 9\cos(\theta) \sin(\varphi)\sin(\theta) \sin(\varphi)\cos(\varphi) \ dS \\
        &= \iint_{S_1} 1 + 3\cos(\theta)\sin^5(\theta)\sin^6(\varphi) + 9\cos(\theta)\sin(\theta) \cos(\varphi)\sin^2(\varphi) \ dS \\
        &= \int_{0}^{\frac{\pi}{2}} \int_0^{2\pi} \left(1 + 3\cos(\theta)\sin^5(\theta)\sin^6(\varphi) + 9\cos(\theta)\sin(\theta) \cos(\varphi)\sin^2(\varphi)\right) \sin(\varphi) \ d\theta d\varphi \\
        &= \int_{0}^{\frac{\pi}{2}} \int_0^{2\pi} \sin(\varphi) + 3\cos(\theta)\sin^5(\theta)\sin^7(\varphi) + 9\cos(\theta)\sin(\theta) \cos(\varphi)\sin^3(\varphi) \ d\theta d\varphi \\
        &= \int_{0}^{\frac{\pi}{2}} \left( \sin(\varphi) \theta - 3\frac{\cos^6(\theta)}{6}\sin^7(\varphi) + 9\frac{\sin^2(\theta)}{2} \cos(\varphi)\sin^3(\varphi)\right)\Biggr|_0^{2\pi} \ d\varphi \\
        &= \int_{0}^{\frac{\pi}{2}} 2\pi\sin(\varphi) + 0 + 0 \ d\varphi \\
        &= 2\pi\int_{0}^{\frac{\pi}{2}} \sin(\varphi) \ d\varphi \\
        &= - 2\pi \cos(\varphi) \Big|_0^{\frac{\pi}{2}} \\
        &= - 2\pi \cos\left(\frac{\pi}{2}\right) + 2\pi \cos(0) \\
        &= 2\pi
    \end{align*}
    
    Now, recall polar coordinates:
    \begin{align}
        x &= r \cos(\theta) & y &= r \sin(\theta) & dA &= r \ dr d\theta
    \end{align}
    
    Then the flux across the disk is given by:
    \begin{align*}
        \iint_{S_2} \vb{F} \cdot d\vb{S} &= \iint_{S_2} \vb{F} \cdot \vb{\hat{n}} \ dS \\
        &= \iint_{S_2} (x + 3y^5) \vb{i} + (y + 10xz) \vb{j} + (z - xy) \vb{k} \cdot (-\vb{k}) \ dS \\
        &= \iint_{S_2} -(z - xy) \ dS
    \end{align*}
    where \(\vb{\hat{n}} = -\vb{k}\) is the outward-pointing normal relative to the solid half ball.
    
    Since the surface \(S_2\) lies on the \(z = 0\) plane, we must evaluate \(\vb{F}\) at \(z = 0\):
    \begin{align*}
        \iint_{S_2} \vb{F} \cdot d\vb{S} &= \iint_{S_2} -z + xy \ dS \\
        &= \iint_{S_2} xy \ dS \\
        &= \iint_{D_2} xy \ dA \\
        &= \int_{0}^{2\pi} \int_0^1 r\cos(\theta) r\sin(\theta) r \ drd\theta \\
        &= \int_{0}^{2\pi} \int_0^1 r^3\cos(\theta) \sin(\theta) \ drd\theta \\
        &= \int_{0}^{2\pi} \frac{r^4}{4} \Biggr|_0^1 \cos(\theta) \sin(\theta) \ d\theta \\
        &= \frac{1}{4} \int_{0}^{2\pi} \cos(\theta) \sin(\theta) \ d\theta \\
        &= \frac{1}{4} \frac{\sin^2(\theta)}{2}\Biggr|_{0}^{2\pi} \\
        &= 0
    \end{align*}
    
    Then the total flux of \(\vb{F}\) across \(S\) is given by:
    \begin{align*}
        \iint_S \vb{F} \cdot d\vb{S} &= \iint_{S_1} \vb{F} \cdot d\vb{S} + \iint_{S_2} \vb{F} \cdot d\vb{S} \\
        &= 2\pi + 0 \\
        &= 2\pi
    \end{align*}
\end{solution}

\begin{solution}
    \textit{(Divergence Theorem)}
    
    Let the region \(W\) denote the solid half ball:
    \begin{align}
        W &= \{(x, y, z) \in \mathbb{R}^3 \ | \ x^2 + y^2 + z^2 \leq 1, \ z \geq 0\}
    \end{align} 
    
    It follows then that
    \begin{align}
        S &= \partial W
    \end{align}

    Observe that \(\vb{F}\) has no singularities in \(W\).

    Then by the Divergence Theorem:
    \begin{align*}
        \iint_S \vb{F} \cdot \ d\vb{S} &= \iiint_W \nabla \cdot \vb{F} \ dV \\
        &= \iiint_W \nabla \cdot \left(x + 3y^5, y + 10xz, z - xy\right) \ dV \\
        &= \iiint_W 1 + 1 + 1 \ dV \\
        &= 3 \iiint_W dV \\
        &= 3 \left(\frac{1}{2}\frac{4\pi}{3}\right) \\
        &= 2\pi
    \end{align*}
    where we used the fact that the volume of a sphere of radius \(R\) is equal to \(\frac{4\pi R^3}{3}\).
\end{solution}