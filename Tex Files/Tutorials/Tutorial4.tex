\subsection{Section 3.5}

\begin{tcolorbox}[
        title={Problem 18},
        valign=center,
        nobeforeafter,
        colframe=gray!95!black
    ]
Is it possible to solve the system of equations:
\begin{align}
    \begin{cases}
        xy^2 + xzu + yv^2 &= 3 \\
    u^3yz + 2xv - u^2v^2 &= 2
    \end{cases}
\end{align}

for \(u(x, y, z)\), \(v(x, y, z)\) near
\((x, y, z) = (1, 1, 1)\), \((u, v) = (1, 1)\)? \\

Compute \(\frac{\partial v}{\partial y}\) at \((x, y, z) = (1, 1, 1)\).
\end{tcolorbox}

\begin{solution}

Let:
\begin{align}
    F_1 &= xy^2 + xzu + yv^2 - 3 & F_2 &= u^3yz + 2xv - u^2v^2 - 2
\end{align}

In order to determine whether the system of equations can be solved, we must compute the determinant of the following matrix at the point \((x, y, z, u, v) = (1, 1, 1, 1, 1)\):
\begin{align}
    \begin{pmatrix}
        \frac{\partial F_1}{\partial u} & \frac{\partial F_1}{\partial v} \\
        \frac{\partial F_2}{\partial u} & \frac{\partial F_2}{\partial v}
    \end{pmatrix}
\end{align}

We first compute the partial derivatives of \(F_1\) and \(F_2\) with respect to \(u\) and \(v\) at \((x, y, z, u, v) = (1, 1, 1, 1, 1)\):
\begin{align}
    \frac{\partial F_1}{\partial u} &= xz & \frac{\partial F_1}{\partial v} &= 2yv \\
    &= 1 & &= 2 \\
    \frac{\partial F_2}{\partial u} &= 3u^2yz - 2uv^2 & \frac{\partial F_2}{\partial v} &= 2x - 2u^2v \\
    &= 1 & &= 0
\end{align}

Then:
\begin{align*}
    \begin{vmatrix}
        \frac{\partial F_1}{\partial u} & \frac{\partial F_1}{\partial v} \\
        \frac{\partial F_2}{\partial u} & \frac{\partial F_2}{\partial v}
    \end{vmatrix} &=
    \begin{vmatrix}
        1 & 2 \\
        1 & 0 
    \end{vmatrix} \\
    &= -2 \\
    &\neq 0
\end{align*}

By the General Implicit Function Theorem, there exist unique (smooth) functions \(u(x,y,z)\) and \(v(x,y,z)\) near the point \((x,y,z,u,v) = (1,1,1,1,1)\). The system of equations is therefore solvable.

The partial derivatives of \(u\) and \(v\) may also be computed by implicit differentiation.

We now wish to compute \(\frac{\partial v}{\partial y}\) at \((x, y, z) = (1, 1, 1)\). 

We must implicitly differentiate both \(F_1\) and \(F_2\) with respect to \(y\) at \((x, y, z, u, v) = (1,1,1,1,1)\):
\begin{align*}
    \frac{\partial F_1}{\partial y} &= \frac{\partial}{\partial y}\left(xy^2 + xzu + yv^2 - 3\right) & \frac{\partial F_2}{\partial y} &= \frac{\partial}{\partial y}\left(u^3yz + 2xv - u^2v^2 - 2\right) \\
    0 &= 2xy + xz\frac{\partial u}{\partial y} + v^2 + 2yv\frac{\partial v}{\partial y} & 0 &= 3u^2\frac{\partial u}{\partial y}yz + u^3z + 2x\frac{\partial v}{\partial y} - 2u\frac{\partial u}{\partial y}v^2 - 2u^2v \frac{\partial v}{\partial y} \\
    0 &= 2 + \frac{\partial u}{\partial y} + 1 + 2\frac{\partial v}{\partial y} & 0 &= 3\frac{\partial u}{\partial y} + 1 + 2\frac{\partial v}{\partial y} - 2\frac{\partial u}{\partial y} - 2\frac{\partial v}{\partial y} \\
    0 &= 3 + \frac{\partial u}{\partial y} + 2\frac{\partial v}{\partial y} & 0 &= \frac{\partial u}{\partial y} + 1
\end{align*}

We now solve for \(\frac{\partial v}{\partial y}\) by algebraic manipulation:
\begin{align*}
    0 &= 3 + \frac{\partial u}{\partial y} + 2\frac{\partial v}{\partial y} \\
    0 &= 3 + (-1) + 2\frac{\partial v}{\partial y} \\
    0 &= 2 + 2\frac{\partial v}{\partial y} \\
    -1 &= \frac{\partial v}{\partial y}
\end{align*}
\end{solution}

\subsection{Section 4.3}

\begin{tcolorbox}[
        title={Problem 20},
        valign=center,
        nobeforeafter,
        colframe=gray!95!black
    ]
Show that \(\vb{c}(t) = (a \cos(t) - b \sin(t) , a \sin(t) + b \cos(t))\) is a flow line for \(\vb{F}(x, y) = (-y, x)\) for all real values of \(a\) and \(b\).
\end{tcolorbox}

\begin{proof}

Recall that a path \(\vb{c}(t)\) is a flow line for a vector field \(\vb{F}\) if:
\begin{align}
    \vb{c}'(t) &= \vb{F}(\vb{c}(t))
\end{align}

We first compute the derivative of \(\vb{c}(t)\):
\begin{align*}
    \vb{c}'(t) &= \frac{d}{dt}(a \cos(t) - b \sin(t) , a \sin(t) + b \cos(t)) \\
    &= (-a \sin(t) - b \cos(t) , a \cos(t) - b \sin(t))
\end{align*}

We now evaluate \(\vb{F}\) along the path \(\vb{c}(t)\):
\begin{align*}
    \vb{F}(\vb{c}(t)) &= \left(- \left(a\sin(t) + b\cos(t)\right), a \cos(t) - b\sin(t)\right) \\
    &= \left(- a\sin(t) - b\cos(t), a \cos(t) - b\sin(t)\right)
\end{align*}

By observation, we find that:
\begin{align}
    \vb{c}'(t) &= \vb{F}(\vb{c}(t))
\end{align}

Therefore, \(\vb{c}(t)\) is a flow line for \(\vb{F}\).
\end{proof}

\begin{tcolorbox}[
        title={Problem 21 (a)},
        valign=center,
        nobeforeafter,
        colframe=gray!95!black
    ]
Let \(\vb{F}(x, y, z) = (yz, xz, xy)\). \\

Find a function \(f : \mathbb{R}^3 \rightarrow \mathbb{R}\) such that \(\vb{F} = \nabla f\) .
\end{tcolorbox}

The following, systematic method can be used to solve harder problems. 

\begin{solution}
    Recall the definition of the gradient:
    \begin{align}
        \nabla f &= \left(\frac{\partial f}{\partial x}, \frac{\partial f}{\partial y}, \frac{\partial f}{\partial z}\right)
    \end{align}
    
    We wish to find a function \(f\) such that:
    \begin{align}
        \left(\frac{\partial f}{\partial x}, \frac{\partial f}{\partial y}, \frac{\partial f}{\partial z}\right) &= (yz, xz, xy) \label{tut4 gradient}
    \end{align}
    
    We integrate the first component with respect to \(x\):
    \begin{align*}
        \int \frac{\partial f}{\partial x} \ dx &= \int yz \ dx \\
        f(x, y, z) &= xyz + g(y, z)
    \end{align*}
    where \(g(y, z)\) is our constant of integration with respect to \(x\). 
    
    We differentiate the function \(f\) with respect to \(y\):
    \begin{align*}
        \frac{\partial f}{\partial y} &= \frac{\partial}{\partial y}\left(xyz + g(y, z)\right) \\
        xz &= xz + \frac{\partial g}{\partial y}
    \end{align*}
    
    In order to satisfy Equation \eqref{tut4 gradient}, we require \(\frac{\partial g}{\partial y} = 0\). This implies that \(g(y, z)\) is a function of \(z\) only: \(g(y, z) =  h(z)\) for some function \(h(z)\).
    
    Then:
    \begin{align*}
        f(x, y, z) &= xyz + h(z)
    \end{align*}
    
    We differentiate the function \(f\) with respect to \(z\):
    \begin{align*}
        \frac{\partial f}{\partial z} &= \frac{\partial}{\partial z}\left(xyz + h(z)\right) \\
        xy &= xy + \frac{\partial h}{\partial z}
    \end{align*}
    
    In order to satisfy Equation \eqref{tut4 gradient}, we require \(\frac{\partial h}{\partial z} = 0\). This implies that \(h(z)\) is a constant: \(h(z) =  C\) for some \(C \in \mathbb{R}\).
    
    Then:
    \begin{align*}
        f(x, y, z) &= xyz + C
    \end{align*}
    
    Therefore, we have found a function \(f\) such that:
    \begin{align}
        \nabla f &= (yz, xz, xy)
    \end{align}
    
\end{solution}

\subsection{Section 4.4}

\begin{tcolorbox}[
        title={Problem 34 (a)},
        valign=center,
        nobeforeafter,
        colframe=gray!95!black
    ]
Show that \(\vb{F} = (x^2 + y^2)\vb{i} - 2xy\vb{j}\) is not a gradient field.
\end{tcolorbox}

\begin{proof}
    Recall that gradient fields are curl free. \\
    
    We compute the curl of \(\vb{F}\):
    \begin{align*}
        \nabla \times \vb{F} &= 
        \begin{vmatrix}
            \vb{i} & \vb{j} & \vb{k} \\
            \frac{\partial}{\partial x} & \frac{\partial}{\partial y} & \frac{\partial}{\partial z} \\
            x^2 + y^2 & -2xy & 0
        \end{vmatrix} \\
        &= (0 + 0)\vb{i} + (0 + 0)\vb{j} + (-2y - 2y)\vb{k} \\
        &= -4y \vb{k} \\
        &\neq \vb{0}
    \end{align*}
    
    Therefore, \(\vb{F}\) is not a gradient field.
\end{proof}